\documentclass[12pt,a4paper]{article}
\usepackage[utf8]{inputenc}
\usepackage[vietnamese]{babel}
\usepackage{array}
\usepackage{longtable}
\usepackage{booktabs}
\usepackage{geometry}
\usepackage{amssymb}
\usepackage{multirow}
\usepackage{tabularx}
\usepackage{tikz}
\usetikzlibrary{shapes, arrows, positioning, calc, fit}

\geometry{left=2cm, right=2cm, top=2cm, bottom=2cm}

\title{\textbf{CRC CARDS\\HỆ THỐNG QUẢN LÝ TUYỂN DỤNG}}
\author{}
\date{}

\begin{document}

\maketitle

\section{Xây dựng các CRC Cards}

% ================== CRC 1: RecruitmentPlan ==================
\begin{table}[h!]
\centering
\caption{CRC cho class RecruitmentPlan (Kế hoạch tuyển dụng)}
\renewcommand{\arraystretch}{1.2} 

\begin{tabular}{|>{\arraybackslash}p{3cm}|>{\arraybackslash}p{6cm}|>{\arraybackslash}p{6cm}|} 

\hline
\textbf{Class}

        RecruitmentPlan
        
        (Kế hoạch tuyển dụng)
        
&   
        \textbf{Responsibilities}
        
        - Thu nhận kế hoạch tuyển dụng từ Trưởng đơn vị
        
        - Gửi kế hoạch cho TP. TC-CB xem xét
        
        - Lưu trữ thông tin kế hoạch
        
& 
        \textbf{Collaborators}
        
        PersonnelManager
        
        Rector
        
        JobPosting
        
\\ \hline
&
        \textbf{Attributes}

        - Mã kế hoạch (id)
        
        - Chức danh cần tuyển
        
        - Số lượng cần tuyển 
        
        - Công việc đảm nhiệm 
        
        - Yêu cầu năng lực, trình độ
        
        - Thời gian tuyển dụng
        
        - Trạng thái (status)

& 
        \textbf{Relationships}

        Generalization (a-kind-of):

        Aggregation (has-part): JobPosting

        Other Associations: PersonnelManager, Rector
    
\\ \hline
\end{tabular}
\end{table}

% ================== CRC 2: User ==================
\begin{table}[h!]
\centering
\caption{CRC cho class User (Người dùng hệ thống)}
\renewcommand{\arraystretch}{1.2} 

\begin{tabular}{|>{\arraybackslash}p{3cm}|>{\arraybackslash}p{6cm}|>{\arraybackslash}p{6cm}|} 

\hline
\textbf{Class}

        User
        
        (Người dùng hệ thống)
        
&   
        \textbf{Responsibilities}
        
        - Xác thực đăng nhập hệ thống
        
        - Quản lý thông tin cá nhân
        
        - Phân quyền truy cập
        
& 
        \textbf{Collaborators}
        
        PersonnelManager
        
        Rector
        
        Candidate
        
\\ \hline
&
        \textbf{Attributes}

        - Mã người dùng (id)
        
        - Tên đăng nhập (username)
        
        - Mật khẩu (password)
        
        - Email
        
        - Họ tên (fullName)
        
        - Số điện thoại (phone)
        
        - Vai trò (role)
        
        - Trạng thái (active)

& 
        \textbf{Relationships}

        Generalization (a-kind-of):

        Aggregation (has-part):

        Other Associations: PersonnelManager, Rector, Candidate
    
\\ \hline
\end{tabular}
\end{table}

\newpage

% ================== CRC 3: PersonnelManager ==================
\begin{table}[h!]
\centering
\caption{CRC cho class PersonnelManager (Cán bộ nhân sự)}
\renewcommand{\arraystretch}{1.2} 

\begin{tabular}{|>{\arraybackslash}p{3cm}|>{\arraybackslash}p{6cm}|>{\arraybackslash}p{6cm}|} 

\hline
\textbf{Class}

        PersonnelManager
        
        (Cán bộ nhân sự - TP. TC-CB)
        
&   
        \textbf{Responsibilities}
        
        - Lập kế hoạch tuyển dụng
        
        - Đăng thông báo tuyển dụng
        
        - Thu thập và sàng lọc hồ sơ
        
        - Thông báo kết quả tuyển dụng
        
& 
        \textbf{Collaborators}
        
        RecruitmentPlan
        
        JobPosting
        
        Application
        
        RecruitmentNotification
        
\\ \hline
&
        \textbf{Attributes}

        - Mã cán bộ (id)
        
        - Mã nhân viên (employeeCode)
        
        - Phòng ban (department)
        
        - Chức vụ (position)

& 
        \textbf{Relationships}

        Generalization (a-kind-of): User

        Aggregation (has-part):

        Other Associations: RecruitmentPlan, JobPosting, Application
    
\\ \hline
\end{tabular}
\end{table}

% ================== CRC 4: Rector ==================
\begin{table}[h!]
\centering
\caption{CRC cho class Rector (Hiệu trưởng)}
\renewcommand{\arraystretch}{1.2} 

\begin{tabular}{|>{\arraybackslash}p{3cm}|>{\arraybackslash}p{6cm}|>{\arraybackslash}p{6cm}|} 

\hline
\textbf{Class}

        Rector
        
        (Hiệu trưởng)
        
&   
        \textbf{Responsibilities}
        
        - Xem xét kế hoạch tuyển dụng
        
        - Phê duyệt kế hoạch tuyển dụng
        
        - Từ chối kế hoạch (nếu cần)
        
& 
        \textbf{Collaborators}
        
        RecruitmentPlan
        
        PersonnelManager
        
\\ \hline
&
        \textbf{Attributes}

        - Mã hiệu trưởng (id)
        
        - Chức danh (title)
        
        - Nhiệm kỳ (term)

& 
        \textbf{Relationships}

        Generalization (a-kind-of): User

        Aggregation (has-part):

        Other Associations: RecruitmentPlan
    
\\ \hline
\end{tabular}
\end{table}

\newpage

% ================== CRC 5: Candidate ==================
\begin{table}[h!]
\centering
\caption{CRC cho class Candidate (Ứng viên)}
\renewcommand{\arraystretch}{1.2} 

\begin{tabular}{|>{\arraybackslash}p{3cm}|>{\arraybackslash}p{6cm}|>{\arraybackslash}p{6cm}|} 

\hline
\textbf{Class}

        Candidate
        
        (Ứng viên)
        
&   
        \textbf{Responsibilities}
        
        - Xem thông báo tuyển dụng
        
        - Nộp hồ sơ ứng tuyển
        
        - Cập nhật thông tin cá nhân
        
        - Xem kết quả ứng tuyển
        
& 
        \textbf{Collaborators}
        
        Application
        
        JobPosting
        
        RecruitmentNotification
        
\\ \hline
&
        \textbf{Attributes}

        - Mã ứng viên (id)
        
        - Họ tên (fullName)
        
        - Email
        
        - Số điện thoại (phone)
        
        - Địa chỉ (address)
        
        - Ngày sinh (dateOfBirth)
        
        - Giới tính (gender)
        
        - Trình độ học vấn (education)

& 
        \textbf{Relationships}

        Generalization (a-kind-of): User

        Aggregation (has-part): Application

        Other Associations: JobPosting, RecruitmentNotification
    
\\ \hline
\end{tabular}
\end{table}

% ================== CRC 6: Application ==================
\begin{table}[h!]
\centering
\caption{CRC cho class Application (Đơn ứng tuyển)}
\renewcommand{\arraystretch}{1.2} 

\begin{tabular}{|>{\arraybackslash}p{3cm}|>{\arraybackslash}p{6cm}|>{\arraybackslash}p{6cm}|} 

\hline
\textbf{Class}

        Application
        
        (Đơn ứng tuyển)
        
&   
        \textbf{Responsibilities}
        
        - Lưu trữ thông tin đơn ứng tuyển
        
        - Quản lý tệp đính kèm (CV)
        
        - Cập nhật trạng thái xét duyệt
        
& 
        \textbf{Collaborators}
        
        Candidate
        
        JobPosting
        
        PersonnelManager
        
        RecruitmentResult
        
\\ \hline
&
        \textbf{Attributes}

        - Mã đơn (id)
        
        - Ngày nộp (submittedDate)
        
        - CV đính kèm (cvUrl)
        
        - Thư giới thiệu (coverLetter)
        
        - Trạng thái (status)
        
        - Ghi chú (notes)

& 
        \textbf{Relationships}

        Generalization (a-kind-of):

        Aggregation (has-part):

        Other Associations: Candidate, JobPosting, PersonnelManager
    
\\ \hline
\end{tabular}
\end{table}

\newpage

% ================== CRC 7: JobPosting ==================
\begin{table}[h!]
\centering
\caption{CRC cho class JobPosting (Tin tuyển dụng)}
\renewcommand{\arraystretch}{1.2} 

\begin{tabular}{|>{\arraybackslash}p{3cm}|>{\arraybackslash}p{6cm}|>{\arraybackslash}p{6cm}|} 

\hline
\textbf{Class}

        JobPosting
        
        (Tin tuyển dụng)
        
&   
        \textbf{Responsibilities}
        
        - Lưu trữ thông tin tin tuyển dụng
        
        - Hiển thị công khai cho ứng viên
        
        - Quản lý thời hạn nộp hồ sơ
        
& 
        \textbf{Collaborators}
        
        RecruitmentPlan
        
        JobPosition
        
        Application
        
        PersonnelManager
        
\\ \hline
&
        \textbf{Attributes}

        - Mã tin (id)
        
        - Tiêu đề (title)
        
        - Mô tả công việc (description)
        
        - Yêu cầu (requirements)
        
        - Mức lương (salary)
        
        - Địa điểm làm việc (location)
        
        - Ngày đăng (postedDate)
        
        - Hạn nộp (deadline)
        
        - Trạng thái (status)

& 
        \textbf{Relationships}

        Generalization (a-kind-of):

        Aggregation (has-part): Application

        Other Associations: RecruitmentPlan, JobPosition
    
\\ \hline
\end{tabular}
\end{table}

% ================== CRC 8: RecruitmentNotification ==================
\begin{table}[h!]
\centering
\caption{CRC cho class RecruitmentNotification (Thông báo tuyển dụng)}
\renewcommand{\arraystretch}{1.2} 

\begin{tabular}{|>{\arraybackslash}p{3cm}|>{\arraybackslash}p{6cm}|>{\arraybackslash}p{6cm}|} 

\hline
\textbf{Class}

        RecruitmentNotification
        
        (Thông báo tuyển dụng)
        
&   
        \textbf{Responsibilities}
        
        - Lưu trữ nội dung thông báo
        
        - Công bố thông báo công khai
        
        - Gửi email thông báo cho ứng viên
        
& 
        \textbf{Collaborators}
        
        PersonnelManager
        
        Candidate
        
        RecruitmentPlan
        
\\ \hline
&
        \textbf{Attributes}

        - Mã thông báo (id)
        
        - Tiêu đề (title)
        
        - Nội dung (content)
        
        - Loại thông báo (type)
        
        - Ngày tạo (createdDate)
        
        - Ngày công bố (publishedDate)
        
        - Trạng thái (status)

& 
        \textbf{Relationships}

        Generalization (a-kind-of):

        Aggregation (has-part):

        Other Associations: PersonnelManager, Candidate, RecruitmentPlan
    
\\ \hline
\end{tabular}
\end{table}

\newpage

% ================== CRC 9: JobPosition ==================
\begin{table}[h!]
\centering
\caption{CRC cho class JobPosition (Vị trí công việc)}
\renewcommand{\arraystretch}{1.2} 

\begin{tabular}{|>{\arraybackslash}p{3cm}|>{\arraybackslash}p{6cm}|>{\arraybackslash}p{6cm}|} 

\hline
\textbf{Class}

        JobPosition
        
        (Vị trí công việc)
        
&   
        \textbf{Responsibilities}
        
        - Lưu trữ thông tin vị trí công việc
        
        - Phân loại các tin tuyển dụng
        
& 
        \textbf{Collaborators}
        
        JobPosting
        
        RecruitmentPlan
        
\\ \hline
&
        \textbf{Attributes}

        - Mã vị trí (id)
        
        - Tên vị trí (name)
        
        - Mô tả (description)
        
        - Phòng ban (department)

& 
        \textbf{Relationships}

        Generalization (a-kind-of):

        Aggregation (has-part):

        Other Associations: JobPosting, RecruitmentPlan
    
\\ \hline
\end{tabular}
\end{table}

% ================== CRC 10: RecruitmentResult ==================
\begin{table}[h!]
\centering
\caption{CRC cho class RecruitmentResult (Kết quả tuyển dụng)}
\renewcommand{\arraystretch}{1.2} 

\begin{tabular}{|>{\arraybackslash}p{3cm}|>{\arraybackslash}p{6cm}|>{\arraybackslash}p{6cm}|} 

\hline
\textbf{Class}

        RecruitmentResult
        
        (Kết quả tuyển dụng)
        
&   
        \textbf{Responsibilities}
        
        - Lưu trữ kết quả xét duyệt
        
        - Cập nhật trạng thái trúng tuyển
        
        - Cung cấp dữ liệu cho thông báo kết quả
        
& 
        \textbf{Collaborators}
        
        Application
        
        Candidate
        
        RecruitmentNotification
        
\\ \hline
&
        \textbf{Attributes}

        - Mã kết quả (id)
        
        - Kết quả (result): PASSED/FAILED
        
        - Điểm đánh giá (score)
        
        - Nhận xét (comments)
        
        - Ngày công bố (announcedDate)

& 
        \textbf{Relationships}

        Generalization (a-kind-of):

        Aggregation (has-part):

        Other Associations: Application, Candidate, RecruitmentNotification
    
\\ \hline
\end{tabular}
\end{table}

\newpage

% ================== BẢNG TỔNG HỢP ==================
\section{Bảng tổng hợp các CRC Cards}

\begin{table}[h!]
\centering
\caption{Tổng hợp các Class trong hệ thống}
\renewcommand{\arraystretch}{1.3}
\begin{tabular}{|c|p{4cm}|p{3.5cm}|p{5cm}|}
\hline
\textbf{STT} & \textbf{Class Name} & \textbf{Mô tả} & \textbf{Collaborators} \\
\hline
1 & RecruitmentPlan & Kế hoạch tuyển dụng & PersonnelManager, Rector, JobPosting \\
\hline
2 & User & Người dùng hệ thống & PersonnelManager, Rector, Candidate \\
\hline
3 & PersonnelManager & Cán bộ nhân sự (TP. TC-CB) & RecruitmentPlan, JobPosting, Application \\
\hline
4 & Rector & Hiệu trưởng & RecruitmentPlan, PersonnelManager \\
\hline
5 & Candidate & Ứng viên & Application, JobPosting, RecruitmentNotification \\
\hline
6 & Application & Đơn ứng tuyển & Candidate, JobPosting, PersonnelManager \\
\hline
7 & JobPosting & Tin tuyển dụng & RecruitmentPlan, JobPosition, Application \\
\hline
8 & RecruitmentNotification & Thông báo tuyển dụng & PersonnelManager, Candidate \\
\hline
9 & JobPosition & Vị trí công việc & JobPosting, RecruitmentPlan \\
\hline
10 & RecruitmentResult & Kết quả tuyển dụng & Application, Candidate, RecruitmentNotification \\
\hline
\end{tabular}
\end{table}

\vspace{1cm}

% ================== MA TRẬN QUAN HỆ LỚP - USECASE ==================
\section{Ma trận quan hệ Lớp - Usecase}

\begin{table}[h]
    \centering
    \caption{Mối quan hệ Lớp - Usecase}
    \label{tab:class_usecase}
    \renewcommand{\arraystretch}{2.0}
    \begin{tabular}{|p{3.5cm}|>{\centering\arraybackslash}p{2cm}|>{\centering\arraybackslash}p{2cm}|>{\centering\arraybackslash}p{2cm}|>{\centering\arraybackslash}p{2cm}|>{\centering\arraybackslash}p{2cm}|}
        \hline
        \textbf{Class} & \textbf{Lập kế hoạch tuyển dụng} & \textbf{Duyệt kế hoạch} & \textbf{Thông báo tuyển dụng} & \textbf{Thu thập và lập danh sách hồ sơ dự tuyển} & \textbf{Thông báo kết quả} \\
        \hline
        RecruitmentPlan (Kế hoạch tuyển dụng) & X & X & X &  &  \\
        \hline
        User (Người dùng hệ thống) &  &  &  & X &  \\
        \hline
        PersonnelManager (Cán bộ nhân sự) & X & X & X & X & X \\
        \hline
        Rector (Hiệu trưởng) &  & X &  &  &  \\
        \hline
        Candidate (Ứng viên) &  &  &  & X &  \\
        \hline
        Application (Đơn ứng tuyển) &  &  &  & X & X \\
        \hline
        JobPosting (Tin tuyển dụng) &  &  & X & X &  \\
        \hline
        RecruitmentNotification (Thông báo tuyển dụng) &  &  & X &  & X \\
        \hline
        JobPosition (Vị trí công việc) & X &  & X &  &  \\
        \hline
        RecruitmentResult (Kết quả tuyển dụng) &  &  &  &  & X \\
        \hline
    \end{tabular}
    \renewcommand{\arraystretch}{1.0}
\end{table}

\newpage

% ================== SƠ ĐỒ LỚP (CLASS DIAGRAM) ==================
\section{Sơ đồ lớp (Class Diagram)}

% Định nghĩa style cho class box
\tikzset{
    classbox/.style={
        rectangle, 
        draw=black, 
        thick,
        minimum width=3.5cm,
        text centered,
        font=\small
    },
    classname/.style={
        rectangle,
        draw=black,
        thick,
        fill=yellow!30,
        minimum width=3.5cm,
        minimum height=0.6cm,
        text centered,
        font=\small\bfseries
    },
    attributes/.style={
        rectangle,
        draw=black,
        thick,
        fill=white,
        minimum width=3.5cm,
        text width=3.3cm,
        align=left,
        font=\scriptsize
    },
    methods/.style={
        rectangle,
        draw=black,
        thick,
        fill=white,
        minimum width=3.5cm,
        text width=3.3cm,
        align=left,
        font=\scriptsize
    },
    inheritance/.style={
        ->,
        >=open triangle 60,
        thick
    },
    association/.style={
        ->,
        thick
    },
    aggregation/.style={
        ->,
        thick,
        o-
    }
}

\begin{figure}[h!]
\centering
\resizebox{\textwidth}{!}{
\begin{tikzpicture}[node distance=1.5cm]

% ========== CLASS: User ==========
\node[classname] (user-name) at (0,0) {User};
\node[attributes, below=0pt of user-name] (user-attr) {
    - id: Long\\
    - username: String\\
    - password: String\\
    - email: String\\
    - fullName: String\\
    - phone: String\\
    - role: String\\
    - active: Boolean
};
\node[methods, below=0pt of user-attr] (user-method) {
    + login()\\
    + logout()\\
    + updateProfile()
};

% ========== CLASS: PersonnelManager ==========
\node[classname] (pm-name) at (-5,-6) {PersonnelManager};
\node[attributes, below=0pt of pm-name] (pm-attr) {
    - employeeCode: String\\
    - department: String\\
    - position: String
};
\node[methods, below=0pt of pm-attr] (pm-method) {
    + createPlan()\\
    + postJob()\\
    + collectApplications()\\
    + notifyResults()
};

% ========== CLASS: Rector ==========
\node[classname] (rector-name) at (0,-6) {Rector};
\node[attributes, below=0pt of rector-name] (rector-attr) {
    - title: String\\
    - term: String
};
\node[methods, below=0pt of rector-attr] (rector-method) {
    + reviewPlan()\\
    + approvePlan()\\
    + rejectPlan()
};

% ========== CLASS: Candidate ==========
\node[classname] (cand-name) at (5,-6) {Candidate};
\node[attributes, below=0pt of cand-name] (cand-attr) {
    - address: String\\
    - dateOfBirth: Date\\
    - gender: String\\
    - education: String
};
\node[methods, below=0pt of cand-attr] (cand-method) {
    + viewJobPostings()\\
    + submitApplication()\\
    + viewResults()
};

% ========== CLASS: RecruitmentPlan ==========
\node[classname] (plan-name) at (-7,-12) {RecruitmentPlan};
\node[attributes, below=0pt of plan-name] (plan-attr) {
    - id: Long\\
    - positionTitle: String\\
    - quantity: Integer\\
    - requirements: String\\
    - recruitmentPeriod: Date\\
    - status: String
};
\node[methods, below=0pt of plan-attr] (plan-method) {
    + submit()\\
    + approve()\\
    + reject()
};

% ========== CLASS: JobPosting ==========
\node[classname] (job-name) at (-2,-12) {JobPosting};
\node[attributes, below=0pt of job-name] (job-attr) {
    - id: Long\\
    - title: String\\
    - description: String\\
    - requirements: String\\
    - salary: String\\
    - location: String\\
    - postedDate: Date\\
    - deadline: Date\\
    - status: String
};
\node[methods, below=0pt of job-attr] (job-method) {
    + publish()\\
    + close()\\
    + getApplications()
};

% ========== CLASS: JobPosition ==========
\node[classname] (pos-name) at (-7,-19) {JobPosition};
\node[attributes, below=0pt of pos-name] (pos-attr) {
    - id: Long\\
    - name: String\\
    - description: String\\
    - department: String
};
\node[methods, below=0pt of pos-attr] (pos-method) {
    + getJobPostings()
};

% ========== CLASS: Application ==========
\node[classname] (app-name) at (3,-12) {Application};
\node[attributes, below=0pt of app-name] (app-attr) {
    - id: Long\\
    - submittedDate: Date\\
    - cvUrl: String\\
    - coverLetter: String\\
    - status: String\\
    - notes: String
};
\node[methods, below=0pt of app-attr] (app-method) {
    + submit()\\
    + updateStatus()\\
    + attachCV()
};

% ========== CLASS: RecruitmentNotification ==========
\node[classname] (notif-name) at (8,-12) {RecruitmentNotification};
\node[attributes, below=0pt of notif-name] (notif-attr) {
    - id: Long\\
    - title: String\\
    - content: String\\
    - type: String\\
    - createdDate: Date\\
    - publishedDate: Date\\
    - status: String
};
\node[methods, below=0pt of notif-attr] (notif-method) {
    + publish()\\
    + sendEmail()
};

% ========== CLASS: RecruitmentResult ==========
\node[classname] (result-name) at (5,-19) {RecruitmentResult};
\node[attributes, below=0pt of result-name] (result-attr) {
    - id: Long\\
    - result: String\\
    - score: Double\\
    - comments: String\\
    - announcedDate: Date
};
\node[methods, below=0pt of result-attr] (result-method) {
    + announce()\\
    + updateResult()
};

% ========== RELATIONSHIPS ==========

% Inheritance (Generalization) - User is parent of PersonnelManager, Rector, Candidate
\draw[inheritance] (pm-name.north) -- ++(0,0.5) -| (user-method.south);
\draw[inheritance] (rector-name.north) -- (user-method.south);
\draw[inheritance] (cand-name.north) -- ++(0,0.5) -| (user-method.south);

% PersonnelManager associations
\draw[association] (pm-method.south) -- ++(0,-0.3) -| (plan-name.north);
\draw[association] (pm-method.south) -- (job-name.north);
\draw[association] (pm-attr.east) -- ++(1,0) |- (notif-name.west);

% Rector associations
\draw[association] (rector-method.south) -- ++(0,-0.5) -| ([xshift=0.5cm]plan-name.north);

% Candidate associations
\draw[association] (cand-method.south) -- (app-name.north);
\draw[association] (cand-attr.east) -- ++(0.5,0) |- ([yshift=0.5cm]notif-name.north);

% RecruitmentPlan associations
\draw[aggregation] (plan-method.south) -- (pos-name.north);
\draw[association] ([xshift=1cm]plan-method.south) -- ++(0,-0.5) -| ([xshift=-0.5cm]job-name.south);

% JobPosting associations
\draw[aggregation] (job-method.south) -- ++(0,-0.5) -| (app-name.south);
\draw[association] ([xshift=-0.5cm]job-method.south) -- ++(0,-1) -| (pos-name.east);

% Application associations
\draw[association] (app-method.south) -- (result-name.north);

% RecruitmentNotification associations
\draw[association] (notif-method.south) -- ++(0,-0.5) -| ([xshift=0.5cm]result-name.north);

% RecruitmentResult associations
\draw[association] (result-attr.west) -- ++(-0.5,0) |- ([yshift=-0.5cm]cand-method.south);

\end{tikzpicture}
}
\caption{Sơ đồ lớp (Class Diagram) - Hệ thống Quản lý Tuyển dụng}
\label{fig:class-diagram}
\end{figure}

\newpage

% ================== CHÚ THÍCH SƠ ĐỒ LỚP ==================
\subsection{Chú thích các mối quan hệ trong sơ đồ lớp}

\begin{table}[h!]
\centering
\caption{Các mối quan hệ trong Class Diagram}
\renewcommand{\arraystretch}{1.5}
\begin{tabular}{|c|p{4cm}|p{8cm}|}
\hline
\textbf{STT} & \textbf{Mối quan hệ} & \textbf{Mô tả} \\
\hline
1 & Generalization (Kế thừa) & User $\leftarrow$ PersonnelManager, Rector, Candidate \\
\hline
2 & Association & PersonnelManager $\rightarrow$ RecruitmentPlan (tạo kế hoạch) \\
\hline
3 & Association & PersonnelManager $\rightarrow$ JobPosting (đăng tin) \\
\hline
4 & Association & PersonnelManager $\rightarrow$ RecruitmentNotification (gửi thông báo) \\
\hline
5 & Association & Rector $\rightarrow$ RecruitmentPlan (duyệt kế hoạch) \\
\hline
6 & Association & Candidate $\rightarrow$ Application (nộp đơn) \\
\hline
7 & Association & Candidate $\rightarrow$ RecruitmentNotification (nhận thông báo) \\
\hline
8 & Aggregation & RecruitmentPlan $\diamond\rightarrow$ JobPosition (chứa vị trí) \\
\hline
9 & Association & RecruitmentPlan $\rightarrow$ JobPosting (tạo tin tuyển dụng) \\
\hline
10 & Aggregation & JobPosting $\diamond\rightarrow$ Application (chứa đơn ứng tuyển) \\
\hline
11 & Association & JobPosting $\rightarrow$ JobPosition (thuộc vị trí) \\
\hline
12 & Association & Application $\rightarrow$ RecruitmentResult (có kết quả) \\
\hline
13 & Association & RecruitmentNotification $\rightarrow$ RecruitmentResult (thông báo kết quả) \\
\hline
14 & Association & RecruitmentResult $\rightarrow$ Candidate (thuộc ứng viên) \\
\hline
\end{tabular}
\end{table}

\vspace{0.5cm}

\begin{table}[h!]
\centering
\caption{Ký hiệu trong Class Diagram}
\renewcommand{\arraystretch}{1.5}
\begin{tabular}{|c|p{3cm}|p{8cm}|}
\hline
\textbf{Ký hiệu} & \textbf{Tên} & \textbf{Ý nghĩa} \\
\hline
$\triangleright$ & Generalization & Quan hệ kế thừa (is-a) \\
\hline
$\rightarrow$ & Association & Quan hệ liên kết (uses) \\
\hline
$\diamond\rightarrow$ & Aggregation & Quan hệ bao gồm (has-part), phần tử có thể tồn tại độc lập \\
\hline
$\blacklozenge\rightarrow$ & Composition & Quan hệ hợp thành, phần tử phụ thuộc hoàn toàn \\
\hline
\end{tabular}
\end{table}

\newpage

% ================== SƠ ĐỒ USE CASE ==================
\section{Sơ đồ Use Case (Use Case Diagram)}

\begin{figure}[h!]
\centering
\begin{tikzpicture}[
    actor/.style={rectangle, minimum width=0.8cm, minimum height=1.5cm},
    usecase/.style={ellipse, draw=black, thick, minimum width=3.5cm, minimum height=1.2cm, text centered, text width=3cm, font=\small},
    system/.style={rectangle, draw=black, thick, minimum width=10cm, minimum height=12cm},
    arrow/.style={->, thick},
    include/.style={->, dashed, thick},
    extend/.style={->, dashed, thick}
]

% System boundary
\node[system, label={[font=\large\bfseries]above:Hệ thống Quản lý Tuyển dụng}] (system) at (5,0) {};

% Actors
% Actor: Cán bộ nhân sự (PersonnelManager)
\node at (-2,3) {\includegraphics[width=0.8cm]{example-image}};
\node[below, font=\small\bfseries] at (-2,2.2) {Cán bộ nhân sự};
\node[below, font=\scriptsize] at (-2,1.8) {(PersonnelManager)};

% Actor: Hiệu trưởng (Rector)
\node at (-2,0) {\includegraphics[width=0.8cm]{example-image}};
\node[below, font=\small\bfseries] at (-2,-0.8) {Hiệu trưởng};
\node[below, font=\scriptsize] at (-2,-1.2) {(Rector)};

% Actor: Ứng viên (Candidate)
\node at (-2,-4) {\includegraphics[width=0.8cm]{example-image}};
\node[below, font=\small\bfseries] at (-2,-4.8) {Ứng viên};
\node[below, font=\scriptsize] at (-2,-5.2) {(Candidate)};

% Use Cases
\node[usecase] (uc1) at (5,4) {Lập kế hoạch tuyển dụng};
\node[usecase] (uc2) at (5,1.5) {Duyệt kế hoạch};
\node[usecase] (uc3) at (5,-0.5) {Thông báo tuyển dụng};
\node[usecase] (uc4) at (5,-3) {Thu thập và lập danh sách hồ sơ dự tuyển};
\node[usecase] (uc5) at (5,-5.5) {Thông báo kết quả};

% Relationships - PersonnelManager
\draw[arrow] (-1,3) -- (uc1.west);
\draw[arrow] (-1,2.8) -- (uc2.west);
\draw[arrow] (-1,2.6) -- (uc3.west);
\draw[arrow] (-1,2.4) -- (uc4.west);
\draw[arrow] (-1,2.2) -- (uc5.west);

% Relationships - Rector
\draw[arrow] (-1,0) -- (uc2.west);

% Relationships - Candidate
\draw[arrow] (-1,-4) -- (uc4.west);

% Include relationships
\draw[include] (uc3) -- node[right, font=\scriptsize] {<<include>>} (uc1);
\draw[include] (uc4) -- node[right, font=\scriptsize] {<<include>>} (uc3);
\draw[include] (uc5) -- node[right, font=\scriptsize] {<<include>>} (uc4);

\end{tikzpicture}
\caption{Sơ đồ Use Case - Hệ thống Quản lý Tuyển dụng}
\label{fig:usecase-diagram}
\end{figure}

\newpage

% ================== SƠ ĐỒ USE CASE DẠNG BẢNG ==================
\subsection{Mô tả chi tiết các Use Case}

\begin{table}[h!]
\centering
\caption{Danh sách Use Case và Actor}
\renewcommand{\arraystretch}{1.5}
\begin{tabular}{|c|p{5cm}|p{4cm}|p{4cm}|}
\hline
\textbf{STT} & \textbf{Use Case} & \textbf{Actor chính} & \textbf{Classes liên quan} \\
\hline
1 & Lập kế hoạch tuyển dụng & PersonnelManager & RecruitmentPlan, JobPosition \\
\hline
2 & Duyệt kế hoạch & Rector, PersonnelManager & RecruitmentPlan \\
\hline
3 & Thông báo tuyển dụng & PersonnelManager & JobPosting, RecruitmentNotification, JobPosition \\
\hline
4 & Thu thập và lập danh sách hồ sơ dự tuyển & PersonnelManager, Candidate & Application, User, JobPosting, Candidate \\
\hline
5 & Thông báo kết quả & PersonnelManager & RecruitmentResult, RecruitmentNotification, Application \\
\hline
\end{tabular}
\end{table}

\newpage

% ================== SƠ ĐỒ TUẦN TỰ (SEQUENCE DIAGRAMS) ==================
\section{Sơ đồ tuần tự (Sequence Diagrams)}

% ===== Sequence Diagram 1: Lập kế hoạch tuyển dụng =====
\subsection{Sơ đồ tuần tự: Lập kế hoạch tuyển dụng}

\begin{figure}[h!]
\centering
\begin{tikzpicture}[
    actor/.style={rectangle, draw=black, thick, minimum width=2cm, minimum height=0.8cm, fill=yellow!20},
    object/.style={rectangle, draw=black, thick, minimum width=2.5cm, minimum height=0.8cm, fill=blue!10},
    lifeline/.style={dashed, thick},
    message/.style={->, thick},
    return/.style={->, dashed, thick}
]

% Actors and Objects
\node[actor] (pm) at (0,0) {PersonnelManager};
\node[object] (plan) at (4,0) {RecruitmentPlan};
\node[object] (pos) at (8,0) {JobPosition};

% Lifelines
\draw[lifeline] (0,-0.4) -- (0,-8);
\draw[lifeline] (4,-0.4) -- (4,-8);
\draw[lifeline] (8,-0.4) -- (8,-8);

% Messages
\draw[message] (0,-1) -- node[above, font=\scriptsize] {1: createPlan()} (4,-1);
\draw[message] (4,-1.8) -- node[above, font=\scriptsize] {2: getPositions()} (8,-1.8);
\draw[return] (8,-2.5) -- node[above, font=\scriptsize] {3: positionList} (4,-2.5);
\draw[message] (0,-3.2) -- node[above, font=\scriptsize] {4: setPositionTitle()} (4,-3.2);
\draw[message] (0,-4) -- node[above, font=\scriptsize] {5: setQuantity()} (4,-4);
\draw[message] (0,-4.8) -- node[above, font=\scriptsize] {6: setRequirements()} (4,-4.8);
\draw[message] (0,-5.6) -- node[above, font=\scriptsize] {7: submit()} (4,-5.6);
\draw[return] (4,-6.4) -- node[above, font=\scriptsize] {8: planCreated} (0,-6.4);

% Activation boxes
\draw[fill=white] (-0.15,-0.8) rectangle (0.15,-6.6);
\draw[fill=white] (3.85,-0.8) rectangle (4.15,-6.6);
\draw[fill=white] (7.85,-1.6) rectangle (8.15,-2.7);

\end{tikzpicture}
\caption{Sơ đồ tuần tự - Lập kế hoạch tuyển dụng}
\label{fig:seq-create-plan}
\end{figure}

\newpage

% ===== Sequence Diagram 2: Duyệt kế hoạch =====
\subsection{Sơ đồ tuần tự: Duyệt kế hoạch}

\begin{figure}[h!]
\centering
\begin{tikzpicture}[
    actor/.style={rectangle, draw=black, thick, minimum width=2cm, minimum height=0.8cm, fill=yellow!20},
    object/.style={rectangle, draw=black, thick, minimum width=2.5cm, minimum height=0.8cm, fill=blue!10},
    lifeline/.style={dashed, thick},
    message/.style={->, thick},
    return/.style={->, dashed, thick}
]

% Actors and Objects
\node[actor] (rector) at (0,0) {Rector};
\node[object] (plan) at (4,0) {RecruitmentPlan};
\node[actor] (pm) at (8,0) {PersonnelManager};

% Lifelines
\draw[lifeline] (0,-0.4) -- (0,-8);
\draw[lifeline] (4,-0.4) -- (4,-8);
\draw[lifeline] (8,-0.4) -- (8,-8);

% Messages
\draw[message] (0,-1) -- node[above, font=\scriptsize] {1: reviewPlan(planId)} (4,-1);
\draw[return] (4,-1.8) -- node[above, font=\scriptsize] {2: planDetails} (0,-1.8);
\draw[message] (0,-2.8) -- node[above, font=\scriptsize] {3: approvePlan()} (4,-2.8);
\draw[message] (4,-3.6) -- node[above, font=\scriptsize] {4: setStatus(APPROVED)} (4,-4.2);
\draw[message] (4,-4.8) -- node[above, font=\scriptsize] {5: notifyApproval()} (8,-4.8);
\draw[return] (4,-5.6) -- node[above, font=\scriptsize] {6: approved} (0,-5.6);

% Activation boxes
\draw[fill=white] (-0.15,-0.8) rectangle (0.15,-5.8);
\draw[fill=white] (3.85,-0.8) rectangle (4.15,-5.8);
\draw[fill=white] (7.85,-4.6) rectangle (8.15,-5.2);

% Alt box for reject
\draw[thick] (-0.5,-6.2) rectangle (6,-7.8);
\node[font=\scriptsize\bfseries] at (0.5,-6.5) {alt [reject]};
\draw[message] (0,-7) -- node[above, font=\scriptsize] {rejectPlan(reason)} (4,-7);

\end{tikzpicture}
\caption{Sơ đồ tuần tự - Duyệt kế hoạch}
\label{fig:seq-approve-plan}
\end{figure}

\newpage

% ===== Sequence Diagram 3: Thông báo tuyển dụng =====
\subsection{Sơ đồ tuần tự: Thông báo tuyển dụng}

\begin{figure}[h!]
\centering
\begin{tikzpicture}[
    actor/.style={rectangle, draw=black, thick, minimum width=2cm, minimum height=0.8cm, fill=yellow!20},
    object/.style={rectangle, draw=black, thick, minimum width=2.2cm, minimum height=0.8cm, fill=blue!10},
    lifeline/.style={dashed, thick},
    message/.style={->, thick},
    return/.style={->, dashed, thick}
]

% Actors and Objects
\node[actor] (pm) at (0,0) {PersonnelMgr};
\node[object] (plan) at (3,0) {RecruitmentPlan};
\node[object] (job) at (6,0) {JobPosting};
\node[object] (notif) at (9.5,0) {Notification};

% Lifelines
\draw[lifeline] (0,-0.4) -- (0,-9);
\draw[lifeline] (3,-0.4) -- (3,-9);
\draw[lifeline] (6,-0.4) -- (6,-9);
\draw[lifeline] (9.5,-0.4) -- (9.5,-9);

% Messages
\draw[message] (0,-1) -- node[above, font=\scriptsize] {1: getApprovedPlan()} (3,-1);
\draw[return] (3,-1.8) -- node[above, font=\scriptsize] {2: planDetails} (0,-1.8);
\draw[message] (0,-2.6) -- node[above, font=\scriptsize] {3: createJobPosting()} (6,-2.6);
\draw[message] (6,-3.4) -- node[above, font=\scriptsize] {4: setDetails()} (6,-4);
\draw[message] (0,-4.6) -- node[above, font=\scriptsize] {5: publish()} (6,-4.6);
\draw[message] (6,-5.4) -- node[above, font=\scriptsize] {6: createNotification()} (9.5,-5.4);
\draw[message] (9.5,-6.2) -- node[above, font=\scriptsize] {7: publish()} (9.5,-6.8);
\draw[return] (6,-7.4) -- node[above, font=\scriptsize] {8: published} (0,-7.4);

% Activation boxes
\draw[fill=white] (-0.15,-0.8) rectangle (0.15,-7.6);
\draw[fill=white] (2.85,-0.8) rectangle (3.15,-2);
\draw[fill=white] (5.85,-2.4) rectangle (6.15,-7.6);
\draw[fill=white] (9.35,-5.2) rectangle (9.65,-7);

\end{tikzpicture}
\caption{Sơ đồ tuần tự - Thông báo tuyển dụng}
\label{fig:seq-job-posting}
\end{figure}

\newpage

% ===== Sequence Diagram 4: Thu thập hồ sơ =====
\subsection{Sơ đồ tuần tự: Thu thập và lập danh sách hồ sơ dự tuyển}

\begin{figure}[h!]
\centering
\begin{tikzpicture}[
    actor/.style={rectangle, draw=black, thick, minimum width=2cm, minimum height=0.8cm, fill=yellow!20},
    object/.style={rectangle, draw=black, thick, minimum width=2cm, minimum height=0.8cm, fill=blue!10},
    lifeline/.style={dashed, thick},
    message/.style={->, thick},
    return/.style={->, dashed, thick}
]

% Actors and Objects
\node[actor] (cand) at (0,0) {Candidate};
\node[object] (job) at (3,0) {JobPosting};
\node[object] (app) at (6,0) {Application};
\node[actor] (pm) at (9.5,0) {PersonnelMgr};

% Lifelines
\draw[lifeline] (0,-0.4) -- (0,-10);
\draw[lifeline] (3,-0.4) -- (3,-10);
\draw[lifeline] (6,-0.4) -- (6,-10);
\draw[lifeline] (9.5,-0.4) -- (9.5,-10);

% Messages
\draw[message] (0,-1) -- node[above, font=\scriptsize] {1: viewJobPostings()} (3,-1);
\draw[return] (3,-1.8) -- node[above, font=\scriptsize] {2: jobList} (0,-1.8);
\draw[message] (0,-2.6) -- node[above, font=\scriptsize] {3: createApplication()} (6,-2.6);
\draw[message] (0,-3.4) -- node[above, font=\scriptsize] {4: attachCV()} (6,-3.4);
\draw[message] (0,-4.2) -- node[above, font=\scriptsize] {5: submit()} (6,-4.2);
\draw[return] (6,-5) -- node[above, font=\scriptsize] {6: submitted} (0,-5);

% PersonnelManager actions
\draw[message] (9.5,-6) -- node[above, font=\scriptsize] {7: getApplications()} (6,-6);
\draw[return] (6,-6.8) -- node[above, font=\scriptsize] {8: applicationList} (9.5,-6.8);
\draw[message] (9.5,-7.6) -- node[above, font=\scriptsize] {9: reviewApplication()} (6,-7.6);
\draw[message] (9.5,-8.4) -- node[above, font=\scriptsize] {10: updateStatus()} (6,-8.4);

% Activation boxes
\draw[fill=white] (-0.15,-0.8) rectangle (0.15,-5.2);
\draw[fill=white] (2.85,-0.8) rectangle (3.15,-2);
\draw[fill=white] (5.85,-2.4) rectangle (6.15,-8.6);
\draw[fill=white] (9.35,-5.8) rectangle (9.65,-8.6);

\end{tikzpicture}
\caption{Sơ đồ tuần tự - Thu thập và lập danh sách hồ sơ dự tuyển}
\label{fig:seq-collect-applications}
\end{figure}

\newpage

% ===== Sequence Diagram 5: Thông báo kết quả =====
\subsection{Sơ đồ tuần tự: Thông báo kết quả}

\begin{figure}[h!]
\centering
\begin{tikzpicture}[
    actor/.style={rectangle, draw=black, thick, minimum width=2cm, minimum height=0.8cm, fill=yellow!20},
    object/.style={rectangle, draw=black, thick, minimum width=2cm, minimum height=0.8cm, fill=blue!10},
    lifeline/.style={dashed, thick},
    message/.style={->, thick},
    return/.style={->, dashed, thick}
]

% Actors and Objects
\node[actor] (pm) at (0,0) {PersonnelMgr};
\node[object] (app) at (3,0) {Application};
\node[object] (result) at (6,0) {Result};
\node[object] (notif) at (9,0) {Notification};
\node[actor] (cand) at (12,0) {Candidate};

% Lifelines
\draw[lifeline] (0,-0.4) -- (0,-10);
\draw[lifeline] (3,-0.4) -- (3,-10);
\draw[lifeline] (6,-0.4) -- (6,-10);
\draw[lifeline] (9,-0.4) -- (9,-10);
\draw[lifeline] (12,-0.4) -- (12,-10);

% Messages
\draw[message] (0,-1) -- node[above, font=\scriptsize] {1: getReviewedApps()} (3,-1);
\draw[return] (3,-1.8) -- node[above, font=\scriptsize] {2: appList} (0,-1.8);
\draw[message] (0,-2.6) -- node[above, font=\scriptsize] {3: createResult()} (6,-2.6);
\draw[message] (0,-3.4) -- node[above, font=\scriptsize] {4: setResult(PASSED)} (6,-3.4);
\draw[message] (0,-4.2) -- node[above, font=\scriptsize] {5: announce()} (6,-4.2);
\draw[message] (6,-5) -- node[above, font=\scriptsize] {6: createNotification()} (9,-5);
\draw[message] (9,-5.8) -- node[above, font=\scriptsize] {7: setContent()} (9,-6.4);
\draw[message] (9,-7) -- node[above, font=\scriptsize] {8: sendEmail()} (12,-7);
\draw[return] (9,-7.8) -- node[above, font=\scriptsize] {9: sent} (6,-7.8);
\draw[return] (6,-8.6) -- node[above, font=\scriptsize] {10: announced} (0,-8.6);

% Activation boxes
\draw[fill=white] (-0.15,-0.8) rectangle (0.15,-8.8);
\draw[fill=white] (2.85,-0.8) rectangle (3.15,-2);
\draw[fill=white] (5.85,-2.4) rectangle (6.15,-8.8);
\draw[fill=white] (8.85,-4.8) rectangle (9.15,-8);
\draw[fill=white] (11.85,-6.8) rectangle (12.15,-7.4);

\end{tikzpicture}
\caption{Sơ đồ tuần tự - Thông báo kết quả}
\label{fig:seq-notify-results}
\end{figure}

\newpage

% ================== SƠ ĐỒ HOẠT ĐỘNG (ACTIVITY DIAGRAM) ==================
\section{Sơ đồ hoạt động (Activity Diagram)}

\subsection{Sơ đồ hoạt động tổng quan quy trình tuyển dụng}

\begin{figure}[h!]
\centering
\begin{tikzpicture}[
    start/.style={circle, draw=black, fill=black, minimum size=0.5cm},
    end/.style={circle, draw=black, thick, minimum size=0.6cm},
    endfill/.style={circle, fill=black, minimum size=0.4cm},
    activity/.style={rectangle, draw=black, thick, rounded corners=5pt, minimum width=3cm, minimum height=0.8cm, text centered, fill=blue!10},
    decision/.style={diamond, draw=black, thick, minimum width=1.5cm, minimum height=1.5cm, aspect=2, fill=yellow!20},
    arrow/.style={->, thick},
    swimlane/.style={rectangle, draw=black, thick, minimum width=4cm}
]

% Swimlanes
\draw[thick] (-1,1) -- (-1,-16);
\draw[thick] (4,1) -- (4,-16);
\draw[thick] (9,1) -- (9,-16);

% Swimlane headers
\node[font=\bfseries] at (1.5,0.5) {PersonnelManager};
\node[font=\bfseries] at (6.5,0.5) {Rector};
\node[font=\bfseries] at (11,0.5) {Candidate};

% Start
\node[start] (start) at (1.5,-0.5) {};

% Activities
\node[activity] (a1) at (1.5,-2) {Lập kế hoạch tuyển dụng};
\node[activity] (a2) at (6.5,-3.5) {Xem xét kế hoạch};
\node[decision] (d1) at (6.5,-5.5) {};
\node[activity] (a3) at (1.5,-5.5) {Chỉnh sửa kế hoạch};
\node[activity] (a4) at (1.5,-7.5) {Đăng thông báo tuyển dụng};
\node[activity] (a5) at (11,-9) {Xem thông báo};
\node[activity] (a6) at (11,-11) {Nộp hồ sơ};
\node[activity] (a7) at (1.5,-11) {Thu thập hồ sơ};
\node[activity] (a8) at (1.5,-13) {Xét duyệt hồ sơ};
\node[activity] (a9) at (1.5,-15) {Thông báo kết quả};
\node[activity] (a10) at (11,-15) {Nhận kết quả};

% End
\node[end] (endouter) at (6.5,-16.5) {};
\node[endfill] at (6.5,-16.5) {};

% Arrows
\draw[arrow] (start) -- (a1);
\draw[arrow] (a1) -- ++(0,-0.5) -| (a2);
\draw[arrow] (a2) -- (d1);
\draw[arrow] (d1) -- node[above, font=\scriptsize] {Từ chối} (a3);
\draw[arrow] (a3) -- ++(0,1) -| (a2);
\draw[arrow] (d1) -- node[right, font=\scriptsize] {Phê duyệt} ++(0,-1) -| (a4);
\draw[arrow] (a4) -- ++(0,-0.5) -| (a5);
\draw[arrow] (a5) -- (a6);
\draw[arrow] (a6) -- (a7);
\draw[arrow] (a7) -- (a8);
\draw[arrow] (a8) -- (a9);
\draw[arrow] (a9) -- (a10);
\draw[arrow] (a10) -- ++(0,-0.5) -| (endouter);

\end{tikzpicture}
\caption{Sơ đồ hoạt động - Quy trình tuyển dụng tổng quan}
\label{fig:activity-diagram}
\end{figure}

\end{document}
